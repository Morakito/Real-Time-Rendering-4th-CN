%%
% Copyright (c) 2017 - 2025, Pascal Wagler;
% Copyright (c) 2014 - 2025, John MacFarlane
%
% All rights reserved.
%
% Redistribution and use in source and binary forms, with or without
% modification, are permitted provided that the following conditions
% are met:
%
% - Redistributions of source code must retain the above copyright
% notice, this list of conditions and the following disclaimer.
%
% - Redistributions in binary form must reproduce the above copyright
% notice, this list of conditions and the following disclaimer in the
% documentation and/or other materials provided with the distribution.
%
% - Neither the name of John MacFarlane nor the names of other
% contributors may be used to endorse or promote products derived
% from this software without specific prior written permission.
%
% THIS SOFTWARE IS PROVIDED BY THE COPYRIGHT HOLDERS AND CONTRIBUTORS
% "AS IS" AND ANY EXPRESS OR IMPLIED WARRANTIES, INCLUDING, BUT NOT
% LIMITED TO, THE IMPLIED WARRANTIES OF MERCHANTABILITY AND FITNESS
% FOR A PARTICULAR PURPOSE ARE DISCLAIMED. IN NO EVENT SHALL THE
% COPYRIGHT OWNER OR CONTRIBUTORS BE LIABLE FOR ANY DIRECT, INDIRECT,
% INCIDENTAL, SPECIAL, EXEMPLARY, OR CONSEQUENTIAL DAMAGES (INCLUDING,
% BUT NOT LIMITED TO, PROCUREMENT OF SUBSTITUTE GOODS OR SERVICES;
% LOSS OF USE, DATA, OR PROFITS; OR BUSINESS INTERRUPTION) HOWEVER
% CAUSED AND ON ANY THEORY OF LIABILITY, WHETHER IN CONTRACT, STRICT
% LIABILITY, OR TORT (INCLUDING NEGLIGENCE OR OTHERWISE) ARISING IN
% ANY WAY OUT OF THE USE OF THIS SOFTWARE, EVEN IF ADVISED OF THE
% POSSIBILITY OF SUCH DAMAGE.
%%

%%
% This is the Eisvogel pandoc LaTeX template.
%
% For usage information and examples visit the official GitHub page:
% https://github.com/Wandmalfarbe/pandoc-latex-template
%%
% Options for packages loaded elsewhere
\PassOptionsToPackage{unicode}{hyperref}
\PassOptionsToPackage{hyphens}{url}
\PassOptionsToPackage{dvipsnames,svgnames,x11names,table}{xcolor}
\documentclass[
  paper=a4,
  ,captions=tableheading
]{scrartcl}
\usepackage{xcolor}
\usepackage{ctex}
\usepackage[margin=2.5cm,includehead=true,includefoot=true,centering,]{geometry}
\usepackage{amsmath,amssymb}


% add backlinks to footnote references, cf. https://tex.stackexchange.com/questions/302266/make-footnote-clickable-both-ways
\usepackage{footnotebackref}
\setcounter{secnumdepth}{-\maxdimen} % remove section numbering
\usepackage{iftex}
\ifPDFTeX
  \usepackage[T1]{fontenc}
  \usepackage[utf8]{inputenc}
  \usepackage{textcomp} % provide euro and other symbols
\else % if luatex or xetex
  \usepackage{unicode-math} % this also loads fontspec
  \defaultfontfeatures{Scale=MatchLowercase}
  \defaultfontfeatures[\rmfamily]{Ligatures=TeX,Scale=1}
\fi
\usepackage{lmodern}
\ifPDFTeX\else
  % xetex/luatex font selection
\fi
% Use upquote if available, for straight quotes in verbatim environments
\IfFileExists{upquote.sty}{\usepackage{upquote}}{}
\IfFileExists{microtype.sty}{% use microtype if available
  \usepackage[]{microtype}
  \UseMicrotypeSet[protrusion]{basicmath} % disable protrusion for tt fonts
}{}

% Use setspace anyway because we change the default line spacing.
% The spacing is changed early to affect the titlepage and the TOC.
\usepackage{setspace}
\setstretch{1.2}

\makeatletter
\@ifundefined{KOMAClassName}{% if non-KOMA class
  \IfFileExists{parskip.sty}{%
    \usepackage{parskip}
  }{% else
    \setlength{\parindent}{0pt}
    \setlength{\parskip}{6pt plus 2pt minus 1pt}}
}{% if KOMA class
  \KOMAoptions{parskip=half}}
\makeatother



\usepackage{longtable,booktabs,array}
\usepackage{calc} % for calculating minipage widths
% Correct order of tables after \paragraph or \subparagraph
\usepackage{etoolbox}
\makeatletter
\patchcmd\longtable{\par}{\if@noskipsec\mbox{}\fi\par}{}{}
\makeatother
% Allow footnotes in longtable head/foot
\IfFileExists{footnotehyper.sty}{\usepackage{footnotehyper}}{\usepackage{footnote}}
\makesavenoteenv{longtable}

\usepackage{graphicx}
\makeatletter
\newsavebox\pandoc@box
\newcommand*\pandocbounded[1]{% scales image to fit in text height/width
  \sbox\pandoc@box{#1}%
  \Gscale@div\@tempa{\textheight}{\dimexpr\ht\pandoc@box+\dp\pandoc@box\relax}%
  \Gscale@div\@tempb{\linewidth}{\wd\pandoc@box}%
  \ifdim\@tempb\p@<\@tempa\p@\let\@tempa\@tempb\fi% select the smaller of both
  \ifdim\@tempa\p@<\p@\scalebox{\@tempa}{\usebox\pandoc@box}%
  \else\usebox{\pandoc@box}%
  \fi%
}
% Set default figure placement to htbp
% Make use of float-package and set default placement for figures to H.
% The option H means 'PUT IT HERE' (as  opposed to the standard h option which means 'You may put it here if you like').
\usepackage{float}
\floatplacement{figure}{H}
\makeatother





\setlength{\emergencystretch}{3em} % prevent overfull lines

\providecommand{\tightlist}{%
  \setlength{\itemsep}{0pt}\setlength{\parskip}{0pt}}




\usepackage{bookmark}
\IfFileExists{xurl.sty}{\usepackage{xurl}}{} % add URL line breaks if available
\urlstyle{same}
\definecolor{default-linkcolor}{HTML}{A50000}
\definecolor{default-filecolor}{HTML}{A50000}
\definecolor{default-citecolor}{HTML}{4077C0}
\definecolor{default-urlcolor}{HTML}{4077C0}

\hypersetup{
  hidelinks,
  breaklinks=true,
  pdfcreator={LaTeX via pandoc with the Eisvogel template}}

\author{}
\date{}


%
% for the background color of the title page
%

%
% break urls
%
\PassOptionsToPackage{hyphens}{url}

%
% When using babel or polyglossia with biblatex, loading csquotes is recommended
% to ensure that quoted texts are typeset according to the rules of your main language.
%
\usepackage{csquotes}

%
% captions
%
\definecolor{caption-color}{HTML}{777777}
\usepackage[font={stretch=1.2}, textfont={color=caption-color}, position=top, skip=4mm, labelfont=bf, singlelinecheck=false, justification=raggedright]{caption}
\setcapindent{0em}

%
% blockquote
%
\definecolor{blockquote-border}{RGB}{221,221,221}
\definecolor{blockquote-text}{RGB}{119,119,119}
\usepackage{mdframed}
\newmdenv[rightline=false,bottomline=false,topline=false,linewidth=3pt,linecolor=blockquote-border,skipabove=\parskip]{customblockquote}
\renewenvironment{quote}{\begin{customblockquote}\list{}{\rightmargin=0em\leftmargin=0em}%
\item\relax\color{blockquote-text}\ignorespaces}{\unskip\unskip\endlist\end{customblockquote}}

%
% Source Sans Pro as the default font family
% Source Code Pro for monospace text
%
% 'default' option sets the default
% font family to Source Sans Pro, not \sfdefault.
%
\ifnum 0\ifxetex 1\fi\ifluatex 1\fi=0 % if pdftex
    \usepackage[default]{sourcesanspro}
  \usepackage{sourcecodepro}
  \else % if not pdftex
    \usepackage[default]{sourcesanspro}
  \usepackage{sourcecodepro}

  % XeLaTeX specific adjustments for straight quotes: https://tex.stackexchange.com/a/354887
  % This issue is already fixed (see https://github.com/silkeh/latex-sourcecodepro/pull/5) but the
  % fix is still unreleased.
  % TODO: Remove this workaround when the new version of sourcecodepro is released on CTAN.
  \ifxetex
    \makeatletter
    \defaultfontfeatures[\ttfamily]
      { Numbers   = \sourcecodepro@figurestyle,
        Scale     = \SourceCodePro@scale,
        Extension = .otf }
    \setmonofont
      [ UprightFont    = *-\sourcecodepro@regstyle,
        ItalicFont     = *-\sourcecodepro@regstyle It,
        BoldFont       = *-\sourcecodepro@boldstyle,
        BoldItalicFont = *-\sourcecodepro@boldstyle It ]
      {SourceCodePro}
    \makeatother
  \fi
  \fi

%
% heading color
%
\definecolor{heading-color}{RGB}{40,40,40}
\addtokomafont{section}{\color{heading-color}}
% When using the classes report, scrreprt, book,
% scrbook or memoir, uncomment the following line.
%\addtokomafont{chapter}{\color{heading-color}}

%
% variables for title, author and date
%
\usepackage{titling}
\title{}
\author{}
\date{}

%
% tables
%

\definecolor{table-row-color}{HTML}{F5F5F5}
\definecolor{table-rule-color}{HTML}{999999}

%\arrayrulecolor{black!40}
\arrayrulecolor{table-rule-color}     % color of \toprule, \midrule, \bottomrule
\setlength\heavyrulewidth{0.3ex}      % thickness of \toprule, \bottomrule
\renewcommand{\arraystretch}{1.3}     % spacing (padding)


%
% remove paragraph indentation
%
\setlength{\parindent}{0pt}
\setlength{\parskip}{6pt plus 2pt minus 1pt}
\setlength{\emergencystretch}{3em}  % prevent overfull lines

%
%
% Listings
%
%


%
% header and footer
%
\usepackage[headsepline,footsepline]{scrlayer-scrpage}

\newpairofpagestyles{eisvogel-header-footer}{
  \clearpairofpagestyles
  \ihead*{}
  \chead*{}
  \ohead*{}
  \ifoot*{}
  \cfoot*{}
  \ofoot*{\thepage}
  \addtokomafont{pageheadfoot}{\upshape}
}
\pagestyle{eisvogel-header-footer}



\begin{document}




\section{Chapter 1 Introduction
简介}\label{chapter-1-introduction-ux7b80ux4ecb}

\subsection{目录}\label{ux76eeux5f55}

\begin{itemize}
\tightlist
\item
  \hyperref[chapter-1-introduction-ux7b80ux4ecb]{Chapter 1 Introduction
  简介}

  \begin{itemize}
  \tightlist
  \item
    \hyperref[ux76eeux5f55]{目录}
  \item
    \hyperref[11-ux5185ux5bb9ux6982ux8ff0]{1.1 内容概述}
  \item
    \hyperref[12-ux7b26ux53f7ux548cux5b9aux4e49]{1.2 符号和定义}

    \begin{itemize}
    \tightlist
    \item
      \hyperref[121-ux6570ux5b66ux7b26ux53f7]{1.2.1 数学符号}
    \item
      \hyperref[122-ux51e0ux4f55ux5b9aux4e49]{1.2.2 几何定义}
    \item
      \hyperref[123-ux7740ux8272]{1.2.3 着色}
    \end{itemize}
  \item
    \hyperref[ux6df1ux5165ux9605ux8bfbux548cux8d44ux6e90]{深入阅读和资源}
  \end{itemize}
\end{itemize}

实时渲染是指在计算机上快速生成图像,它是计算机图形学中互动性最强的领域。屏幕上会显示一张图像,观察者在看到图像之后会做出一些反应和操作,这些反馈紧接着又会对下一张图像的生成产生影响。这个包含反应和渲染的循环会以一个很高的速度发生,以至于观察者根本意识不到自己正在观察一系列相互独立的图像,而是沉浸在这样一个动态的过程中。

通常会使用每秒显示的帧数(frames per
second,FPS)或者赫兹(Hertz,Hz)来衡量图像显示的速率。如果图像以每秒一帧的速率进行显示,那么就几乎没有交互感可言,用户会意识(感知)到每一张新图像的到来,这是一个痛苦的过程。当图像显示速率达到6
FPS左右时,会逐渐开始增加交互性。电子游戏的目标是30,60,72或者更高的FPS,在这样的图像显示速率下,用户可以将注意力集中在自身的行动和反应上。

电影放映机(movie projector)会以24
FPS的速率来进行显示,但是它会使用一个快门系统(shutter
system)来将每帧重复显示2-4次,从而避免画面出现闪烁。这被称为刷新率(refresh
rate),其单位是赫兹(Hz),它和上文中所说的显示速率(display
rate)是两个概念。以刚才的电影放映机为例,一个可以每帧照亮三次的快门,其刷新率为72
Hz。同样的,LCD显示器的刷新率和显示速率也是两个不同的概念。

在显示器上观看以24
FPS出现的图片也许是可以接受的,但是更高的帧率可以有效降低最小反应时间。当显示延迟大于15毫秒的时候,就会对交互的流畅感产生干扰{[}1849{]}。举个例子,头戴的VR显示设备一般需要90
FPS来最小化延迟。

实时渲染也不仅仅只包含交互性,如果渲染速度是唯一的衡量标准的话,那么任何能够快速响应用户指令,并在屏幕上绘制图像的应用程序都符合这个条件。实时渲染通常指的是将三维场景渲染成二维图像。

交互性和三维场景是实时渲染的充分条件,但是第三个元素也已经逐渐成为了其定义的一部分,即图形加速硬件(graphics
acceleration hardware)。许多人都认为,在1996年上市的3Dfx Voodoo
1图形加速卡是消费级显卡的开端{[}408{]}。随着这个市场的快速发展,如今每个电脑、平板和手机中都内置了相应的图形处理器。图1.1和图1.2展示了一些通过硬件加速来实现实时渲染的优秀案例。

\begin{figure}
\centering
\pandocbounded{\includegraphics[keepaspectratio]{images/Chapter-1/202211011124394.png}}
\caption*{图1.1 《极限竞速7》的游戏内画面。}
\end{figure}

\begin{figure}
\centering
\pandocbounded{\includegraphics[keepaspectratio]{images/Chapter-1/202211011128769.png}}
\caption*{图2.2 《巫师3》中的鲍克兰港。}
\end{figure}

图形硬件的进步推动了交互式计算机图形学领域的爆炸式发展。我们将重点关注用于提高渲染速度和图像质量的相关方法,同时也会介绍一些加速算法和图形API的特性,以及它们的局限性。我们不可能对每一个话题都进行深入讨论,因此我们的目标是介绍关键性的概念和术语,介绍该领域中最健壮和最实用的算法,同时提供一些深入学习的方向指引。我们还会尝试提供一些工具,来帮助您更好地理解这个领域,同时希望这份尝试能够对得起您阅读本书所付出的时间和努力。

\subsection{1.1 内容概述}\label{ux5185ux5bb9ux6982ux8ff0}

下面是对各个章节的简要概述。

\textbf{第2章-图形渲染管线(The Graphics Rendering
Pipeline)}。实时渲染的核心是一组操作步骤,它将场景描述作为输入,并将其转换为我们能够看得见的图像。

\textbf{第3章-图形处理单元(The Graphics Processing
Unit)}。现代GPU使用固定功能(fixed-function)单元和可编程单元(programmable)的组合,来实现渲染管线的各个阶段。

\textbf{第4章-变换(Transforms)}。变换是用于控制物体位置、朝向、尺寸、形状以及相机位置、相机视角的基本工具。

\textbf{第5章-着色基础(Shading
Basics)}。我们首先会讨论材质和光源的定义,以及它们如何用于实现所需要的表面外观,无论这个表面是写实的还是风格化的。该章节还会介绍其他与外观表现相关的话题,例如使用抗锯齿、透明度和伽马矫正来获得更高的图像质量。

\textbf{第6章-纹理(Texturing)}。实时渲染中最强大的工具之一,就是能够快速访问图像,并将其显示在表面上。这个过程叫做纹理化,有各种各样来实现它的方法。

\textbf{第7章-阴影(Shadows)}。在场景中添加阴影效果可以增强画面的真实感和表现力。我们会在这个章节中介绍几种目前比较流行的、能够快速计算阴影的方法。

\textbf{第8章-光和颜色(Light and
Color)}。在我们实现基于物理的渲染之前,我们首先需要了解如何对光和颜色进行量化建模。并且在我们执行基于物理的渲染之后,我们还需要将最终结果转换为可以显示的数值,并考虑屏幕属性和观察环境对它的影响。本章节将会介绍以上两个主题。

\textbf{第9章-基于物理的着色( Physically Based
Shading)}。我们将从头开始建立起对基于物理的着色模型的理解。在本章节中,我们首先会从潜在的物理现象开始,介绍一系列包含各种渲染材质的着色模型,最后介绍这些材质的混合方法和过滤方法,从而避免材质出现瑕疵(aliasing)并保持表面外观。

\textbf{第10章-局部光照(Local
Illumination)}。本章节研究了刻画(portray)复杂光源的算法。我们在进行表面着色的时候,需要考虑到这些光线是从某些物理对象中发射出来的,这些物体具有独特的形状。

\textbf{第11章-全局光照(Global
Illumination)}。本章节研究了模拟光线和场景多次相交的算法,这可以大大增强图像的真实感。我们还会研究环境光遮蔽(ambient
occlusion)和定向遮蔽(directional
occlusion),介绍在漫反射表面和镜面表面上渲染全局光照效果的方法,以及一些很有前景的统一方法。

\textbf{第12章-图像空间特效(Image-Space
Effects)}。图形硬件擅长进行高速的图像处理。在本章节中,我们首先会讨论图像过滤技术和重投影(reprojection)技术,然后我们将会介绍几种常见的后处理特效:镜头光晕(lens
flare),动态模糊(motion blur)和景深(depth of field)。

\textbf{第13章-超越多边形(Beyond
Polygons)}。对于描述物体形状而言,三角形并不总是速度最快或者效果最逼真的方法,一些诸如基于图像、点云(point
cloud)、体素(voxel)或者其他样本集合的替代方法,都有着各自的独特优势。

\textbf{第14章-体积和半透明渲染(Volumetric and Translucency
Rendering)}。本章节的重点是体积材质的表示方法,以及体积材质与光线相互作用的理论与实践。它可以模拟的现象有很多,大到大范围的大气效果,小到毛发纤维的光线散射等。

\textbf{第15章-非真实感渲染(Non-Photorealistic
Rendering)}。尝试将一个场景渲染得更加逼真,只是众多渲染方式中的一种,除此之外还有很多其他风格,比如卡通渲染和水彩效果等。同时我们还会对直线和文本生成技术进行讨论。

\textbf{第16章-多边形技术(Polygonal
Techniques)}。几何数据的来源有很多,有时候我们需要对其进行修正,才能更好更快的进行渲染。本章节讨论了有关多边形数据表示和多边形数据压缩的相关内容。

\textbf{第17章-曲线和曲面(Curves and Curved
Surfaces)}。使用一些更加复杂的表面表达方式可以提供很多优势,例如可以在渲染质量和渲染速度之间进行权衡,可以具有更加紧凑的表示以及可以生成更加平滑的表面。

\textbf{第18章-管线优化(Pipeline
Optimization)}。对于一个已经使用了高效算法,并且正在运行的应用程序,我们还可以使用各种优化技术来进一步提高它的运行效率。本章节讨论了如何找应用程序的性能瓶颈(bottleneck)并对其进行处理,以及多线程优化等问题。

\textbf{第19章-加速算法(Acceleration
Algorithms)}。当我们让一个程序成功运行起来之后,下一步就是让它运行得更快。本章节讨论了各种各样的剔除技术(culling),以及层次细节(level
of detail,LOD)等技术。

\textbf{第20章-高效着色(Efficient
Shading)}。场景中的大量光源会严重降低性能表现;在无法确定一个片元最终是否可见之前对其进行着色计算,也是重要的性能开销来源(过度绘制overdraw)。本章节中我们会介绍很多方法,来解决着色过程中可能会出现低效率问题。

\textbf{第21章-虚拟现实和增强现实(Virtual and Augmented
Reality)}。这些领域有着特殊的挑战和技术,例如:如何以一个较高且稳定的帧率,来高效生成逼真的图像。

\textbf{第22章-相交测试技术(Intersection Test
Methods)}。相交测试对于渲染,用户交互和碰撞检测而言十分重要。在本章节中,我们介绍了许多用于几何相交测试的高效算法,并对其进行了深入讨论。

\textbf{第23章-图像硬件(Graphics
Hardware)}。本章节对一些硬件组件进行了重点关注,例如颜色缓冲、深度缓冲、帧缓冲以及其他基本结构类型等。同时提供了一个具有代表性的GPU案例学习。

\textbf{第24章-展望未来(The
Future)}。预测未来的技术发展趋势,以及对读者的建议。

我们还完成了一章有关碰撞检测(Collision
Detection)的内容,以及一章有关实时光线追踪的内容(Real-Time Ray
Tracing),这里限于篇幅,我们将其放在了配套网站\href{http://realtimerendering.com}{realtimerendering.com}上,你可以在这里下载到相关内容,同时网站上还有关于线性代数以及三角学的附录内容。

\subsection{1.2 符号和定义}\label{ux7b26ux53f7ux548cux5b9aux4e49}

首先我们需要解释一下本书中所用到的数学符号。如果你想对本小节或者本书中的术语做更加深入的了解,你可以在\href{http://realtimerendering.com}{realtimerendering.com}上找到我们的线性代数附录。

\subsubsection{1.2.1 数学符号}\label{ux6570ux5b66ux7b26ux53f7}

表1.1中总结了大部分我们将要用到的数学符号,这里我们将对其中的一些概念进行详细描述。

请注意表格中的规则也有一些例外,这主要是因为着色方程中所使用的符号,在相关文献中已经非常完善且统一了,例如\(L\)代表辐射度(radiance),\(E\)代表辐照度(irradiance),\(\sigma_s\)代表散射系数(scattering
coefficient)等。

角度和标量都是取自于\(\mathbb{R}\)(实数集),即它们都是实数。向量和点使用粗体的小写字母进行表示,其各个分量的表示如下:

\[
\mathbf{v} = \left(  \begin{array}{c} v_x \\ v_y \\ v_z \end{array} \right)
\]

该向量以列向量的形式给出,这种表达形式在计算机图形学中被广泛使用。在本书中的某些地方我们会使用行向量\((v_x, v_y, v_z)\)来表示向量或者点,之所以不使用形式更加正确的\((v_x, v_y, v_z)^T\),只是因为前者阅读起来更加容易。

\begin{longtable}[]{@{}
  >{\raggedright\arraybackslash}p{(\linewidth - 4\tabcolsep) * \real{0.1615}}
  >{\raggedright\arraybackslash}p{(\linewidth - 4\tabcolsep) * \real{0.1231}}
  >{\raggedright\arraybackslash}p{(\linewidth - 4\tabcolsep) * \real{0.7154}}@{}}
\toprule\noalign{}
\begin{minipage}[b]{\linewidth}\raggedright
\textbf{类型}
\end{minipage} & \begin{minipage}[b]{\linewidth}\raggedright
\textbf{数学标记}
\end{minipage} & \begin{minipage}[b]{\linewidth}\raggedright
\textbf{例子}
\end{minipage} \\
\midrule\noalign{}
\endhead
\bottomrule\noalign{}
\endlastfoot
角度(angle) & 小写希腊字母 &
\(\alpha_i, \phi, \rho, \eta, \gamma_{242}, \theta\) \\
标量(scalar) & 小写斜体 & \(a, b, t, u_k, v, w_{ij}\) \\
向量,点(vector,point) & 小写粗体 &
\(\mathbf{a}, \mathbf{u}, \mathbf{v}_s, \mathbf{h}(\rho), \mathbf{h}_z\) \\
矩阵(matrix) & 大写粗体 &
\(\mathbf{T(t)}, \mathbf{X}, \mathbf{R}_x(\rho),\) \\
平面(plane) & \(\pi\):一个向量和一个标量 &
\(\pi :\mathbf{n} \cdot \mathbf{x} + d = 0,  \pi_1 :\mathbf{n}_1 \cdot \mathbf{x} + d_1 = 0\) \\
三角形(triangle) & \(\triangle\)+三个顶点 &
\(\triangle \mathbf{v}_0 \mathbf{v}_1 \mathbf{v}_2, \triangle \mathbf{cba}\) \\
线段(line segment) & 两个顶点 &
\(\mathbf{uv}, \mathbf{a}_i \mathbf{b}_j\) \\
几何实体(geometry entity) & 大写斜体 & \(A_{\text{OBB}}, T, B_{\text{AABB}}\) \\
\end{longtable}

使用齐次(homogeneous)坐标表示法,一个坐标可以使用四个值来进行表示,即\(\mathbf{v} =(v_x \quad v_y \quad v_z \quad v_w)^T\),其中\(\mathbf{v} =(v_x \quad v_y \quad v_z \quad 0)^T\)代表一个向量,\(\mathbf{v} =(v_x \quad v_y \quad v_z \quad 1)^T\)代表一个点。有时我们会使用只包含三个分量的向量或者点,我们会尽量避免关于使用何种表示类型的歧义。对于矩阵运算而言,使用相同符号形式的点和向量是十分有用的,更多内容详见第4章中有关变换的部分。在某些算法中,使用数字索引来代替\(x, y, z\)下标会很方便,例如\(\mathbf{v} =(v_0 \quad v_1 \quad v_2)^T\)。所有这些有关向量和点的符号规则,同样也适用于只包含两个分量的向量,在二维向量的情况中,我们会直接跳过向量的第三个分量。

矩阵值得我们多进行一些解释。常用的矩阵尺寸包括\(2 \times 2\),\(3 \times 3\),\(4 \times 4\),这里我们将以\(3 \times 3\)矩阵\(\mathbf{M}\)为例,来回顾矩阵的访问方式,其他尺寸矩阵的操作也类似。矩阵\(\mathbf{M}\)的(标量)元素记为\(m_{ij}, 0 \le (i, j) \le 2\),其中的\(i\)代表该元素所在的行,\(j\)代表该元素所在的列,如方程1.1所示:

\[
\mathbf{M} = 
\left( \begin{array}{ccc} 
m_{00} & m_{01} & m_{02} \\ 
m_{10} & m_{11} & m_{12} \\
m_{20} & m_{21} & m_{22}  
\end{array} \right) \tag{1.1}
\]

方程1.2中的符号也代表一个\(3 \times 3\)矩阵,这种表达形式用于从矩阵\(\mathbf{M}\)中分离向量:\(\mathbf{m}_{,j}\)代表第\(j\)个列向量;\(\mathbf{m}_{i,}\)代表第\(i\)个行向量(以列向量形式进行表示)。与向量与点一样,如果使用起来更加方便的话,列向量也可以使用\(x, y, z, w\)来进行索引:

\[
\mathbf{M}=\left(\begin{array}{lll}\mathbf{m}_{, 0} & \mathbf{m}_{, 1} & \mathbf{m}_{, 2}\end{array}\right)=\left(\begin{array}{lll}\mathbf{m}_{x} & \mathbf{m}_{y} & \mathbf{m}_{z}\end{array}\right)=\left(\begin{array}{c}\mathbf{m}_{0,}^{T} \\[2mm] \mathbf{m}_{1,}^{T} \\[2mm] \mathbf{m}_{2,}^{T}\end{array}\right)
\tag{1.2} 
\]

我们使用\(\pi :\mathbf{n} \cdot \mathbf{x} + d = 0\)来表示一个平面,它包含了定义平面所需的数学公式,即平面的法线\(\mathbf{n}\)以及标量\(d\)。其中平面法线是一个描述平面朝向的向量,对于更一般的表面(例如曲面),法线描述了表面上某个特定点的朝向;而对于平面而言,平面上所有点都具有相同的法线。\(\pi\)通常被用作为代表平面的数学符号,平面\(\pi\)会将空间一分为二,其中位于正半空间中的点满足\(\mathbf{n} \cdot \mathbf{x} + d > 0\);位于负半空间中的点满足\(\mathbf{n} \cdot \mathbf{x} + d < 0\)。剩下所有的点都位于平面\(\pi\)上。

一个三角形可以使用三个顶点\(\mathbf{v}_0, \mathbf{v}_1, \mathbf{v}_2\)来进行定义,记为\(\triangle \mathbf{v}_0 \mathbf{v}_1 \mathbf{v}_2\)。

表1.2展示了其他的一些数学运算符及其符号表示,你可以在配套网站\href{http://realtimerendering.com/}{realtimerendering.com}上找到线性代数附录,其中包含了点乘、叉乘、行列式以及模长操作符的相关解释。转置操作符可以将一个列向量转换为一个行向量,反之亦然,这样我们就可以将一个列向量写在一行中,例如\(\mathbf{v} = (v_x \quad v_y \quad v_z)^T\)。表中的第四个操作符在《Graphics Gems IV》{[}735{]}中有详细介绍,这是一个作用于二维向量的一元操作符,它作用于向量\(\mathbf{v} = (v_x \quad v_y)^T\)上,并会生成一个与其垂直的向量,例如\(\mathbf{v}^\perp = (-v_y \quad v_x)^T\)。


\begin{longtable}[]{@{}
  >{\raggedright\arraybackslash}p{(\linewidth - 4\tabcolsep) * \real{0.1615}}
  >{\raggedright\arraybackslash}p{(\linewidth - 4\tabcolsep) * \real{0.1231}}
  >{\raggedright\arraybackslash}p{(\linewidth - 4\tabcolsep) * \real{0.7154}}@{}}
\toprule\noalign{}
\begin{minipage}[b]{\linewidth}\raggedright
\textbf{序号}
\end{minipage} & \begin{minipage}[b]{\linewidth}\raggedright
\textbf{数学标记}
\end{minipage} & \begin{minipage}[b]{\linewidth}\raggedright
\textbf{说明}
\end{minipage} \\
\midrule\noalign{}
\endhead
\bottomrule\noalign{}
\endlastfoot
1 & \(\cdot\) & 点乘 \\
2 & \(\times\) & 叉乘 \\
3 & \(\mathbf{v}^T\) & 向量\(\mathbf{v}\)的转置 \\
4 & \(^\perp\) & 一元操作符,垂直点乘操作符 \\
5 & \(\vert \cdot \vert\) & 矩阵的行列式 \\
6 & \(\vert \cdot \vert\) & 标量的绝对值 \\
7 & \(\Vert \cdot \Vert\) & 范数(长度和模长) \\
8 & \(x^+\) & 将x的最小值限制在0 \\
9 & \(x^\mp\) & 将\(x\)限制在0到1之间 \\
10 & \(n!\) & 阶乘 \\
11 & \(\left( \begin{array}{c} n \\ k  \end{array}\right)\) &
二项式系数 \\
\end{longtable}

我们使用\(\vert a \vert\)来表示标量\(a\)的绝对值,使用\(\vert \mathbf{A} \vert\)来表示矩阵\(\mathbf{A}\)的行列式。有时我们还会使用\(\vert \mathbf{A} \vert = \vert \mathbf{a \quad b \quad c} \vert = \det(\mathbf{a,b,c})\)这种表示方式,其中\(\mathbf{a,b,c}\)分别是矩阵\(\mathbf{A}\)的列向量。

第8和第9个操作符是限制操作符(clamp),它在着色计算中经常使用。操作符8会将输入值的负数部分限制到0:

\[
x^+ = \left \{ 
\begin{array}{ll}
x & \text{if} \ x > 0, \\
0 & \text{otherwise},
\end{array} 
\right.
\tag{1.3} 
\]

操作符9则会将输入值限制在0到1之间:

\[
x^\mp = \left \{ 
\begin{array}{ll}
1, & \text{if} \ x \ge 1, \\
x, & \text{if} \ 0 < x < 1, \\
0, & \text{otherwise}
\end{array} 
\right.
\tag{1.4} 
\]

操作符10是阶乘(factorial)操作符,其定义如下所示,请注意\(0! = 1:\)

\[
n! = n(n-1)(n-2) \cdots 3 \cdot 2 \cdot 1 \tag{1.5}
\]

操作符11是组合数,也叫做二项式系数,其定义方程1.6:

\[
\left( \begin{array}{c} n \\ k  \end{array}\right) = 
\frac{n!}{k! (n-k)!} \tag{1.6}
\]

除此之外,我们一般将\(x = 0\),\(y = 0\),\(z = 0\)这三个平面叫做坐标平面(coordinate
planes)或者轴对齐平面(axis-aligned planes)。将 \[
\mathbf{e}_x = \left( \begin{array}{c} 1 \\ 0 \\ 0  \end{array} \right),
\mathbf{e}_y = \left( \begin{array}{c} 0 \\ 1 \\ 0  \end{array} \right),
\mathbf{e}_z = \left( \begin{array}{c} 0 \\ 0 \\ 1  \end{array} \right)
\] 叫做主轴(main axes)或者主方向(main
direction);或者分别叫做\(x\)轴,\(y\)轴和\(z\)轴。这组向量通常也会被称为标准基(standard
basis)。除了特殊说明之外,我们将会使用标准正交基(即由相互垂直的单位向量所组成的基底)。

我们将同时包含\(a,b\),以及两者之间所有数字的范围区间记为\([a,b]\)。如果我们只想要\(a,b\)之间的数字,而不想要\(a,b\)本身的话,那么我们可以将其记为\((a, b)\)。我们也可以将开闭区间进行组合使用,例如:\([a,b)\)代表包括\(a\)在内,但是不包括\(b\)在内的,\(a,b\)之间的所有数字。

\begin{longtable}[]{@{}lll@{}}
\toprule\noalign{}
序号 & 函数 & 描述 \\
\midrule\noalign{}
\endhead
\bottomrule\noalign{}
\endlastfoot
1 & \(\mathsf{atan2}(y, x)\) & 二元反正切函数 \\
2 & \(\log(n)\) & \(n\)的自然对数 \\
\end{longtable}

\(\mathsf{atan2}(y, x)\)是一个C语言中的数学函数,它在本文中经常使用,因此值得我们去关注一下。它是数学函数\(\arctan(x)\)的一个拓展,它俩的主要区别在于\(-\frac{\pi}{2} < \arctan(x) < \frac{\pi}{2}\),而\(-\pi  \le \mathsf{atan2}(y, x) \le \pi\);并且后者包含一个额外的参数输入。这个函数的常见应用是用来计算\(\arctan(y/x)\),当\(x = 0\)时,分母就为0了。而拥有两个参数的\(\mathsf{atan2}(y, x)\)则可以避免这一点。

在本书中,
\(\log(n)\)始终代表了自然对数,即\(\log_e(n)\),而不是以10为底的对数\(\log_{10}(n)\)。

颜色使用一个三维向量来进行表示,例如\((red,green,blue)\),其中每个分量的范围都是\([0,1]\)。

\subsubsection{1.2.2 几何定义}\label{ux51e0ux4f55ux5b9aux4e49}

几乎所有图形硬件使用的渲染图元(primitive,也叫做drawing
primitives)都是点、线和三角形。

\begin{quote}
我们所知道的唯二例外就是Pixel-Planes{[}502{]},它可以绘制球体;以及NVIDIA
NV1芯片,它可以绘制椭球体。
\end{quote}

在本书中,我们会将一个几何实体(geometric
entities)的集合称作为模型(model)或者物体(object)。场景(scene)是指环境中所有待渲染模型的集合,同时场景中还包含了材质信息,灯光信息,以及观察信息等。

这里的物体可以是一辆车,一栋建筑甚至是一条直线。在实际中,一个物体中包含了一系列的渲染图元,但是也有例外,物体也可以是其他更加高级的几何表现形式,例如Bezier
曲线(Bezier curves)、Bezier 曲面或者是细分曲面(subdivision
surface)。同时,一个物体也可以同时包含其他的物体,例如一辆车包含了四个车门以及四个轮子等。

\subsubsection{1.2.3 着色}\label{ux7740ux8272}

按照约定俗成的计算机图形学惯例,本书中的``着色(shading)''和``着色器(shader)''以及相关的派生词,常常被用来指向两个相关但是完全不同的概念:一个是计算机生成的视觉外观,例如:``着色模型(shading
model)'',``着色方程(shading equation)'',``卡通渲染(toon
shading)''等;另一个是渲染系统中的可编程组件,例如:``顶点着色器(vertex
shader)'',``着色器语言(shading
language)''等。在这两种不同的情况下,你可以通过上下文来推断出它具体指向的含义。

\subsection{深入阅读和资源}\label{ux6df1ux5165ux9605ux8bfbux548cux8d44ux6e90}

我们能够给你提供的、最重要的资源,就是本书的配套网站:\href{http://realtimerendering.com}{realtimerendering.com},其中包含了最新信息的链接以及每章相应的网站。实时渲染的研究领域也是实时变化的,在本书中,我们试图关注那些最基本的概念,以及那些不太可能过时的技术。在这个网站上,我们可以展示与当今软件开发者有关的信息,并且我们有能力将其进行不断更新。

\end{document}
